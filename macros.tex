% tikz stuff and other macros
% Jesse Hamner
% 2013--2017

% make a few color defs:
\definecolor{rltbrightred}{rgb}{1,0,0}
\definecolor{rltred}{rgb}{0.75,0,0}
\definecolor{rltdarkred}{rgb}{0.5,0,0}
\definecolor{rltbrightgreen}{rgb}{0,0.75,0}
\definecolor{rltgreen}{rgb}{0,0.5,0}
\definecolor{rltdarkgreen}{rgb}{0,0,0.25}
\definecolor{rltbrightblue}{rgb}{0,0,1}
\definecolor{rltblue}{rgb}{0,0,0.75}
\definecolor{rltdarkblue}{rgb}{0,0,0.5}
%\definecolor{webred}{rgb}{0.5,.25,0}
\definecolor{webblue}{rgb}{0,0,0.75}
\definecolor{webgreen}{rgb}{0,0.5,0}
\definecolor{lightgray}{gray}{0.9}
\definecolor{medgray}{gray}{0.6}
\definecolor{footlineblue}{rgb}{0.392,0.706,0.941}
\definecolor{darkgreen}{rgb}{0.00,0.50,0.00}
\definecolor{gray50}{gray}{0.50}
\definecolor{gray95}{gray}{0.95}
\definecolor{webred}{rgb}{0.75,0,0.0}

\renewcommand{\thefootnote}{\fnsymbol{footnote}}
\newcommand{\sep}{1mm}
\newcommand{\negsep}{-4mm}
\newcommand{\thisscale}{0.6}

\newcommand{\+}{\item}		% easier than typing \item a lot
%\setlength{\parindent}{0in}	% easier than putting \noindent at every paragraph
\newcommand{\bi}{\begin{itemize}}
\newcommand{\ei}{\end{itemize}}
\newcommand{\bd}{\begin{description}}
\newcommand{\ed}{\end{description}}
\newcommand{\be}{\begin{enumerate}}
\newcommand{\ee}{\end{enumerate}}
\newcommand{\rr}{\raggedright}

\newcommand{\cb}{\cellcolor{black!35}}

\newcommand{\stbox}[1]{
	\begin{center}
	\fbox{
		\begin{minipage}[c]{6.0in}{
		\noindent #1
		}
		\end{minipage}
	}
	\end{center}
}




\newcommand{\ionradius}{0.06} % FIXME should make it possible to change this too

\newcommand{\drawion}[4]{
		\draw [black, very thin] (#1,#2) circle [radius=#3];
		\node[scale=0.3, thick ] at (#1,#2) {#4};
}

\newcommand{\posion}[3]{
		\drawion{#1}{#2}{#3}{$+$}
		}

\newcommand{\negion}[3]{
		\drawion{#1}{#2}{#3}{$-$}
		}

\newcommand{\horizionarray}[4]{
	\newdimen\len
	\len=#1 cm
	\advance\len by -0.4cm
	\drawion{\len}{#2}{\ionradius}{#4}
	\advance\len by 0.15cm
	\drawion{\len}{#2}{\ionradius}{#4}
	\advance\len by 0.15cm
	\drawion{\len}{#2}{\ionradius}{#4}
	\advance\len by 0.2cm
	\drawion{\len}{#2}{\ionradius}{#4}
	\advance\len by 0.15cm
	\drawion{\len}{#2}{\ionradius}{#4}
	\advance\len by 0.15cm
	\drawion{\len}{#2}{\ionradius}{#4}
}

\newcommand{\vertionarray}[4]{
	\newdimen\len
	\len=#2 cm
	\advance\len by -0.4cm
	\drawion{#1}{\len}{\ionradius}{#4}
	\advance\len by 0.15cm
	\drawion{#1}{\len}{\ionradius}{#4}
	\advance\len by 0.15cm
	\drawion{#1}{\len}{\ionradius}{#4}
	\advance\len by 0.2cm
	\drawion{#1}{\len}{\ionradius}{#4}
	\advance\len by 0.15cm
	\drawion{#1}{\len}{\ionradius}{#4}
	\advance\len by 0.15cm
	\drawion{#1}{\len}{\ionradius}{#4}
}

\newcommand{\toninetwo}[3]{
\begin{tikzpicture}

\filldraw[fill=lightgray, draw=black](0.8,0) -- (0.9,0.0) -- (1,0.1) -- (1,0.9) -- (0.9,1.0) -- (0.8,1.0) ;
\filldraw[fill=lightgray, draw=black](0.8,1) arc (90:270:5.5mm and 5mm);

\filldraw[very thin, fill=white](0.8,0.2) circle [radius=0.5mm];
\filldraw[very thin, fill=white](0.8,0.5) circle [radius=0.5mm];
\filldraw[very thin, fill=white](0.8,0.8) circle [radius=0.5mm];

\node[scale=0.7, thick ] at (1.25,0.9) {#1};
\node[scale=0.7, thick ] at (1.25,0.5) {#2};
\node[scale=0.7, thick ] at (1.25,0.1) {#3};
\node[scale=1, thick ] at (0.5,1.8) {TO-92 Package, top view};

\end{tikzpicture}
}

\newcommand{\toeighteen}[3]{
  	\begin{circuitikz}
	\filldraw[thick, fill=lightgray](1,0.5) circle [radius=5mm];

% FIXME is it possible to compute this dynamically? With FP maybe?

% r is 6mm
% x1 + r * cos(theta)   is x
% -0.82 is cos(145), and r cos(theta) is -4.915mm
% x1 + r cos(theta) is  5.085mm
%  
% y1 + r * sin(theta)   is y
% 0.574 is sin(145)
% y1 + 3.44 mm is start value, or 8.44mm
	
	\draw[ draw=black](5.085mm, 8.44mm) arc (145:485:6mm);	

% next draw the tab, at 1 mm wide and long

\draw[rotate around={135:(1.21mm, 6.31mm)}](0mm,0mm) rectangle(2mm,2mm);

	\filldraw[very thin, fill=white](0.85,0.8) circle [radius=0.5mm];
	\filldraw[very thin, fill=white](1.3,0.5) circle [radius=0.5mm];
	\filldraw[very thin, fill=white](0.85,0.2) circle [radius=0.5mm];

	\node[scale=0.7, thick ] at (0.8,1.3) {#1};
	\node[scale=0.7, thick ] at (1.8,0.5) {#2};
	\node[scale=0.7, thick ] at (0.3,0.1) {#3};

	\node[scale=1, thick ] at (0.5,1.8) {TO-18 Package, top view};
	\end{circuitikz}
}


\newcommand{\dpak}[4]{
  	\begin{circuitikz}

% main body: 
	\filldraw[thick, fill=gray](0.0mm,0.0mm) rectangle(10mm,9mm);

% tab:
	\draw[thick](1mm,11mm) to (3mm,12mm)
	to (7mm,12mm)
	to (9mm,11mm)
	to (9mm,10mm)
	to (1mm,10mm)
	to (1mm,11mm);

% scalloped top:
	\filldraw[black,thick, fill=gray]	(0mm,9mm) arc(270:360:1mm) --
	(9mm,10mm) --
	(9mm,10mm) arc(180:270:1mm) --
	(10mm,9mm) --  
	(1mm,9mm) -- 
 	cycle;

% pins:	
	\draw[thick](1mm,0mm) rectangle(2.5mm,-6mm);
	\draw[thick](4.25mm,0mm) rectangle(5.75mm,-3mm);
	\draw[thick](7.5mm,0mm) rectangle(9.0mm,-6mm);

% labels:
	\node[scale=0.7, thick ] at (-0.1,-0.3) {#1};
	\node[scale=0.7, thick ] at (0.5,-0.75) {#2};
	\node[scale=0.7, thick ] at (1.1,-0.3) {#3};
	\node[scale=0.7, thick ] at (0.5,1.4) {#4};
	\node[scale=1, thick ] at (0.5,1.9) {TO-252 Package};
	\end{circuitikz}
}


% P-type MOSFET, e.g.: \totwotwenty{B}{E}{C}{}
\newcommand{\totwotwenty}[4]{
  	\begin{circuitikz}

% main body: 
	\filldraw[thick, fill=gray](0.0mm,0.0mm) rectangle(10mm,9mm);

% tab:
	\draw[thick](0mm,9mm) 
	rectangle(10mm,17mm)
	;
	\draw[thick](5mm,13.5mm) circle(1.75mm);


% pins:	
	\newcommand{\ypll}{-4mm}
	\draw[thick](1mm,0mm) rectangle(2.5mm,\ypll);
	\draw[thick](4.25mm,0mm) rectangle(5.75mm,\ypll);
	\draw[thick](7.5mm,0mm) rectangle(9.0mm,\ypll);
	
% smaller pin descenders:
	\draw[thick](1.5mm,\ypll) rectangle(2.0mm,-9mm);
	\draw[thick](5.0mm,\ypll) rectangle(5.5mm,-9mm);
	\draw[thick](8.0mm,\ypll) rectangle(8.5mm,-9mm);
		

% labels:
	\node[scale=0.7, thick ] at (-0.1,-0.3) {#1};
	\node[scale=0.7, thick ] at (0.5,-1.1) {#2};
	\node[scale=0.7, thick ] at (1.1,-0.3) {#3};
	\node[scale=0.7, thick ] at (0.5,1.4) {#4};
	\node[scale=1, thick ] at (0.5,2.1) {TO-220 Package};
	\end{circuitikz}
}



\newcommand{\msqo}[3]{
\def\mycmd{#3}

\if\mycmd1
	\msq{#1}{#2}{black!65}
\else
	\msq{#1}{#2}{white}
\fi
}

\newcommand{\drawnokiagrid}{
\draw[step=1cm,blue!30,very thin] (0,0) grid (6,8);
\foreach \x in {0,1,2,3,4,5}
	\draw (\x cm, 0pt) -- (\x cm, -3pt) node[anchor=north] {$\x$};
\foreach \y in {1,2,3,4,5,6,7,8}
	\draw (0pt, \y cm) -- (-3pt, \y cm) node[anchor=east] {$\y$};
}

\newcommand{\msq}[3]{
\edef\myx{#1}
\pgfmathparse{\myx+1.0}
\edef\myxtwo{\pgfmathresult}
\edef\myy{#2}
\pgfmathparse{\myy+1.0}
\edef\myytwo{\pgfmathresult}
\fill[#3] (\myx,#2) rectangle (\myxtwo,\myytwo);}

\newcommand{\mcol}[9]{
\msqo{#1}{#2}{#3}
\msqo{#1}{#4}{#5}
\msqo{#1}{#6}{#7}
\msqo{#1}{#8}{#9}
}

\newcommand{\hexlabel}[5]{
\node[rotate=90] at (0.5,-1) {\Large\texttt{#1}};
\node[rotate=90] at (1.5,-1) {\Large\texttt{#2}};
\node[rotate=90] at (2.5,-1) {\Large\texttt{#3}};
\node[rotate=90] at (3.5,-1) {\Large\texttt{#4}};
\node[rotate=90] at (4.5,-1) {\Large\texttt{#5}};
}

\newcommand{\makecube}[2]{
\begin{tikzpicture}[xscale=#2,yscale=#2]
\foreach \x in{0,...,#1}
{   \draw (0,\x ,#1) -- (#1,\x ,#1);
    \draw (\x ,0,#1) -- (\x ,#1,#1);
    \draw (#1,\x ,#1) -- (#1,\x ,0);
    \draw (\x ,#1,#1) -- (\x ,#1,0);
    \draw (#1,0,\x ) -- (#1,#1,\x );
    \draw (0,#1,\x ) -- (#1,#1,\x );
}
\end{tikzpicture}
}

\newcommand{\makeplate}[3]{
\begin{tikzpicture}[xscale=#3,yscale=#3]

\foreach \x in {0,...,#1}
{
	\filldraw[gray95, fill=gray95, line width=0.0cm] (0,#1,0) -- (0,#1,#2) -- (#1,#1,#2) -- (#1,#1, 0) -- (0,#1,0) -- cycle;
	\filldraw[gray50, fill=gray50, line width=0.0cm] (#1,#1,0) -- (#1,#1,#2) -- (#1,0,#2) -- (#1,0, 0) -- (#1,#1,0) -- cycle;
}

\foreach \x in{0,...,#1}
{   \draw (0,\x ,#2) -- (#1,\x ,#2);
    \draw (\x ,0,#2) -- (\x ,#1,#2);
	\draw (#1,\x ,#2 ) -- (#1,\x ,0);
    \draw (\x ,#1,#2 ) -- (\x ,#1,0);
}

\foreach \x in {0,...,#2}
{
    \draw(#1,0,\x  ) -- (#1,#1,\x );
    \draw(0,#1,\x  ) -- (#1,#1,\x );
}

\foreach \x in{1,...,#1}
{
\node[align=left] at (\x-0.5,#1-0.5,#2) {\x};
}

\foreach \x in{#1,...,2}
{
\node[align=left] at (0.5,#1-\x+0.5,#2) {\x};
}

%\draw[webred, thick](0,0,0) -- (0,0,#2);
%\draw[webblue, thick](0,0,0) -- (#1,0,0);
%\draw[darkgreen, thick](0,0,0) -- (0,#1,0);

\end{tikzpicture}
}

\newcommand{\blockline}[2]{
\begin{tikzpicture}[xscale=#2,yscale=#2]

\filldraw[fill=gray95, ultra thin] (0,1,0) -- (0,1,1) -- (#1,1,1) -- (#1,1, 0) -- (0,1,0) -- cycle;
\filldraw[fill=gray50, ultra thin] (#1,1,0) -- (#1,1,1) -- (#1,0,1) -- (#1,0, 0) -- (#1,1,0) -- cycle;

\draw (0,1,0 ) -- (#1,1,0); % top line
\draw (#1,1,1) -- (#1,1,0);
\draw (#1,0,1) -- (#1,1,1);
\draw (0,1, 1) -- (#1,1,1);
\draw (0,0,1 ) -- (#1,0,1);

\foreach \x in{0,...,#1}
{   
    \draw (\x ,0,1) -- (\x ,1,1 ); % straight vertical lines    
    \draw (\x ,1,1) -- (\x ,1,0 );
}

\foreach \x in{1,...,#1}
{
	\node[align=left] at (\x-0.5,0.5,1) {\x};
}

\end{tikzpicture}
}

\newcommand{\makeblank}[1]{
\underline{\makebox[#1][l]{}}
}

\newcommand*\rot{\multicolumn{1}{R{55}{1.2ex}}}% no optional argument here, please!
