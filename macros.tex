% tikz stuff and other macros
% Jesse Hamner
% 2013--2019

% make a few color defs:
\definecolor{rltbrightred}{rgb}{1,0,0}
\definecolor{rltred}{rgb}{0.75,0,0}
\definecolor{rltdarkred}{rgb}{0.5,0,0}
\definecolor{rltbrightgreen}{rgb}{0,0.75,0}
\definecolor{rltgreen}{rgb}{0,0.5,0}
\definecolor{rltdarkgreen}{rgb}{0,0.25,0}
\definecolor{rltbrightblue}{rgb}{0,0,1}
\definecolor{rltblue}{rgb}{0,0,0.75}
\definecolor{rltdarkblue}{rgb}{0,0,0.5}
%\definecolor{webred}{rgb}{0.5,.25,0}
\definecolor{webblue}{rgb}{0,0,0.75}
\definecolor{webgreen}{rgb}{0,0.5,0}
\definecolor{lightgray}{gray}{0.9}
\definecolor{medgray}{gray}{0.6}
\definecolor{footlineblue}{rgb}{0.392,0.706,0.941}
\definecolor{darkgreen}{rgb}{0.00,0.50,0.00}
\definecolor{gray50}{gray}{0.50}
\definecolor{gray95}{gray}{0.95}
\definecolor{webred}{rgb}{0.75,0,0.0}

\renewcommand{\thefootnote}{\fnsymbol{footnote}}
\newcommand{\sep}{1mm}
\newcommand{\negsep}{-4mm}
\newcommand{\widesep}{7mm}
\newcommand{\thisscale}{0.6}
\newcommand{\grr}{\rowcolor{gray!15}}

\newcommand{\+}{\item}		% easier than typing \item a lot
\newcommand{\bi}{\begin{itemize}}
\newcommand{\ei}{\end{itemize}}
\newcommand{\bd}{\begin{description}}
\newcommand{\ed}{\end{description}}
\newcommand{\be}{\begin{enumerate}}
\newcommand{\ee}{\end{enumerate}}
\newcommand{\rr}{\raggedright}

\newcommand{\cb}{\cellcolor{black!35}}

\newcommand{\ckb}{\item[$\square$]} % list checkboxes (requires marvosym)

% A command that makes a uniform box around a paragraph. 
% It's a minipage environment, which means it doesn't support some commands.
\newcommand{\stbox}[2]{
	\begin{center}
	\fbox{
		\begin{minipage}[c]{#1}{
		\noindent #2
		}
		\end{minipage}
	}
	\end{center}
}


% A series of macros that enable programmatic of drawing stylized
% ions, protons, or electrons.
% Supports,  e.g., richer drawings of capacitors or
% current flowing in wires:

\newcommand{\ionradius}{0.06} % TODO should make it possible to change this too

\newcommand{\drawion}[4]{
		\draw [black, very thin] (#1,#2) circle [radius=#3];
		\node[scale=0.3, thick ] at (#1,#2) {#4};
}

\newcommand{\posion}[3]{
		\drawion{#1}{#2}{#3}{$+$}
		}

\newcommand{\negion}[3]{
		\drawion{#1}{#2}{#3}{$-$}
		}

\newcommand{\horizionarray}[4]{
	\newdimen\len
	\len=#1 cm
	\advance\len by -0.4cm
	\drawion{\len}{#2}{\ionradius}{#4}
	\advance\len by 0.15cm
	\drawion{\len}{#2}{\ionradius}{#4}
	\advance\len by 0.15cm
	\drawion{\len}{#2}{\ionradius}{#4}
	\advance\len by 0.2cm
	\drawion{\len}{#2}{\ionradius}{#4}
	\advance\len by 0.15cm
	\drawion{\len}{#2}{\ionradius}{#4}
	\advance\len by 0.15cm
	\drawion{\len}{#2}{\ionradius}{#4}
}

\newcommand{\vertionarray}[4]{
	\newdimen\len
	\len=#2 cm
	\advance\len by -0.4cm
	\drawion{#1}{\len}{\ionradius}{#4}
	\advance\len by 0.15cm
	\drawion{#1}{\len}{\ionradius}{#4}
	\advance\len by 0.15cm
	\drawion{#1}{\len}{\ionradius}{#4}
	\advance\len by 0.2cm
	\drawion{#1}{\len}{\ionradius}{#4}
	\advance\len by 0.15cm
	\drawion{#1}{\len}{\ionradius}{#4}
	\advance\len by 0.15cm
	\drawion{#1}{\len}{\ionradius}{#4}
}



% Nokia 84 x 48 pixel LCD screen (usually model number 5110 or 3310)
\newcommand{\drawnokiagrid}{
\draw[step=1cm,blue!30,very thin] (0,0) grid (6,8);
\foreach \x in {0,1,2,3,4,5}
	\draw (\x cm, 0pt) -- (\x cm, -3pt) node[anchor=north] {$\x$};
\foreach \y in {1,2,3,4,5,6,7,8}
	\draw (0pt, \y cm) -- (-3pt, \y cm) node[anchor=east] {$\y$};
}


% part of the Nokia grid set of commands
\newcommand{\msqo}[3]{
\def\mycmd{#3}

\if\mycmd1
	\msq{#1}{#2}{black!65}
\else
	\msq{#1}{#2}{white}
\fi
}


% part of the Nokia grid set of commands
\newcommand{\msq}[3]{
\edef\myx{#1}
\pgfmathparse{\myx+1.0}
\edef\myxtwo{\pgfmathresult}
\edef\myy{#2}
\pgfmathparse{\myy+1.0}
\edef\myytwo{\pgfmathresult}
\fill[#3] (\myx,#2) rectangle (\myxtwo,\myytwo);}


% part of the Nokia grid set of commands
\newcommand{\mcol}[9]{
\msqo{#1}{#2}{#3}
\msqo{#1}{#4}{#5}
\msqo{#1}{#6}{#7}
\msqo{#1}{#8}{#9}
}


% part of the Nokia grid set of commands
\newcommand{\hexlabel}[5]{
\node[rotate=90] at (0.5,-1) {\Large\texttt{#1}};
\node[rotate=90] at (1.5,-1) {\Large\texttt{#2}};
\node[rotate=90] at (2.5,-1) {\Large\texttt{#3}};
\node[rotate=90] at (3.5,-1) {\Large\texttt{#4}};
\node[rotate=90] at (4.5,-1) {\Large\texttt{#5}};
}



% drawing cubes using TikZ:
\newcommand{\makecube}[2]{
\begin{tikzpicture}[xscale=#2,yscale=#2]
\foreach \x in{0,...,#1}
{   \draw (0,\x ,#1) -- (#1,\x ,#1);
    \draw (\x ,0,#1) -- (\x ,#1,#1);
    \draw (#1,\x ,#1) -- (#1,\x ,0);
    \draw (\x ,#1,#1) -- (\x ,#1,0);
    \draw (#1,0,\x ) -- (#1,#1,\x );
    \draw (0,#1,\x ) -- (#1,#1,\x );
}
\end{tikzpicture}
}


% Draw faux-3D squares or cubes:
\newcommand{\makeplate}[3]{
\begin{tikzpicture}[xscale=#3,yscale=#3]

\foreach \x in {0,...,#1}
{
	\filldraw[gray95, fill=gray95, line width=0.0cm] (0,#1,0) -- (0,#1,#2) -- (#1,#1,#2) -- (#1,#1, 0) -- (0,#1,0) -- cycle;
	\filldraw[gray50, fill=gray50, line width=0.0cm] (#1,#1,0) -- (#1,#1,#2) -- (#1,0,#2) -- (#1,0, 0) -- (#1,#1,0) -- cycle;
}

\foreach \x in{0,...,#1}
{   \draw (0,\x ,#2) -- (#1,\x ,#2);
    \draw (\x ,0,#2) -- (\x ,#1,#2);
	\draw (#1,\x ,#2 ) -- (#1,\x ,0);
    \draw (\x ,#1,#2 ) -- (\x ,#1,0);
}

\foreach \x in {0,...,#2}
{
    \draw(#1,0,\x  ) -- (#1,#1,\x );
    \draw(0,#1,\x  ) -- (#1,#1,\x );
}

\foreach \x in{1,...,#1}
{
\node[align=left] at (\x-0.5,#1-0.5,#2) {\x};
}

\foreach \x in{#1,...,2}
{
\node[align=left] at (0.5,#1-\x+0.5,#2) {\x};
}

%\draw[webred, thick](0,0,0) -- (0,0,#2);
%\draw[webblue, thick](0,0,0) -- (#1,0,0);
%\draw[darkgreen, thick](0,0,0) -- (0,#1,0);

\end{tikzpicture}
}


% Draw a line of faux-3D shaded blocks using TikZ:
\newcommand{\blockline}[2]{
\begin{tikzpicture}[xscale=#2,yscale=#2]

\filldraw[fill=gray95, ultra thin] (0,1,0) -- (0,1,1) -- (#1,1,1) -- (#1,1, 0) -- (0,1,0) -- cycle;
\filldraw[fill=gray50, ultra thin] (#1,1,0) -- (#1,1,1) -- (#1,0,1) -- (#1,0, 0) -- (#1,1,0) -- cycle;

\draw (0,1,0 ) -- (#1,1,0); % top line
\draw (#1,1,1) -- (#1,1,0);
\draw (#1,0,1) -- (#1,1,1);
\draw (0,1, 1) -- (#1,1,1);
\draw (0,0,1 ) -- (#1,0,1);

\foreach \x in{0,...,#1}
{   
    \draw (\x ,0,1) -- (\x ,1,1 ); % straight vertical lines    
    \draw (\x ,1,1) -- (\x ,1,0 );
}

\foreach \x in{1,...,#1}
{
	\node[align=left] at (\x-0.5,0.5,1) {\x};
}

\end{tikzpicture}
}


% make a blank line under which, e.g., someone could write an unknown answer:
\newcommand{\makeblank}[1]{
\underline{\makebox[#1][l]{}}
}


% tilt the text atop a table row at an angle:
\newcommand*\rot{\multicolumn{1}{R{55}{1.2ex}}}% no optional argument here, please!
