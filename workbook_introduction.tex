\section{Introduction}

This workshop is going to help you learn about computers -- but not how to \emph{use} computers. Rather, when we are finished, you will understand a little bit about how and why computers really work. To get to that goal, you'll need to learn some new topics in math, and learn about electricity. You'll also need to learn about basic \emph{electronic components}, too. These are the small pieces that enable electricity to do all the things we can make electricity do, like light our homes, play music on the radio, find directions with a GPS, run a gasoline engine, or show you a web page.

Computers are wonderful, multipurpose machines. But computers can't add or subtract the way you do. They can't read the way you do. They don't think. Really,  \emph{they only do math.} Everything computers do is controlled by a series of instructions that \emph{all boil down to math and logic,} even if it looks like magic. The way computers do all the things they do can always be described as a  series of instructions. Each of these actions can be performed by some combination of special circuit types, and each circuit is made up of basic electronic components. You can learn enough about each basic component to understand what it does, without needing to do a bunch of math. And, you can understand all the individual instructions, even if you don't want to build your own computer!

Ultimately this workshop will make it possible for you to make, from the most basic electronic components, a simple digital computer that can add two numbers and correctly display the result. Along the way you'll see how each piece works, and how to put the pieces together. Some workshops will also teach you how to solder components, so you can actually \emph{make} some of the building blocks of computers. 

