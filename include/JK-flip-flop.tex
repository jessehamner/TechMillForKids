%\documentclass[12pt]{article}
%\usepackage[left=1in,right=1in,top=1in,bottom=1in]{geometry} 
%\usepackage{xcolor}
%\usepackage{tikz}
%\usepackage[american, EFvoltages, cuteinductors]{circuitikz}
%
%\begin{document}
%\thispagestyle{empty}
%
%\begin{figure}[h!]
%\begin{center}
%
\begin{circuitikz}


% Clock input:
\draw
	(-2cm,2.75cm) node[ocirc](clknode) {} % CLK node
	node[left] {{\color{red}$CLK$}} % CLK label
;


% Inputs and 3-input NAND gates:
\begin{scope}
	\draw
		(0,4.25) 
		node[american nand port, anchor=in 2, number inputs = 3](J){}
		node[ocirc, xshift = -2cm](jnode){}
    	node[xshift = -2cm, yshift = 3mm](){{\color{red}$J$}}
		node[xshift = 0.9cm, yshift = 1.0cm] {{\footnotesize{$NAND_1$}}} 
		(jnode) |-  (J.in 2)
	;

	\draw
		(0,1.75) 
		node[american nand port, anchor=in 1, number inputs = 3](K){}
		node[ocirc, xshift = -2cm, yshift = -4mm](knode){}
    	node[xshift = -2cm, yshift = -1mm](){{\color{red}$K$}}
		node[xshift = 0.9cm, yshift = -1.3cm]{{\footnotesize{$NAND_2$}}} 
		(knode) |-  (K.in 2)
	;
\end{scope}


% Q output:
\draw
    (5,3.97) 
	node[american nand port] (Nand3){}
	node[xshift = -5mm, yshift = 1.0cm]{{\footnotesize{$NAND_3$}}} 
	node[ocirc, xshift=2cm, yshift=0mm](qnode){}
	node[xshift = 2cm, yshift = 3mm](){{\color{red}$Q$}}
	(Nand3.out) to (qnode)
;


% not-Q output:
\draw
	(5,1.65) node[american nand port] (Nand4){}
	node[xshift = -5mm, yshift = -1.0cm] {{\footnotesize{$NAND_4$}}} 
	node[ocirc, xshift=2cm, yshift=0mm](notqnode){}
	node[xshift = 2cm, yshift = 4mm](){{\color{red}$\overline{Q}$}}
	(Nand4.out) to (notqnode)
;


% Nets and labels from the three-input NANDs to the SR circuit:
\draw(J.out) -| (Nand3.in 1);
\draw(K.out) -| (Nand4.in 2);

\draw
	(Nand3.in 1)
	node[yshift = 3mm, xshift = -2mm](nots){{\color{red}$\overline{S}$}}
;

\draw
	(Nand4.in 2)
	node [yshift = -3mm, xshift = -2mm](notr){{\color{red}$\overline{R}$}}
;


% Shaded box surrounding the basic SR Flip-Flop:
\filldraw (5cm,2.8cm) node[minimum size=5cm, draw, fill=blue!40, opacity=0.3]{};


% Clock input line nets:
\draw[blue, thick]
	(-0.5, 2.75) node[circ, color=blue](clknode2){}
	(clknode) to (clknode2)
	(clknode2) |- (J.in 3)
	(clknode2) |- (K.in 1)
;

% Signal / feedback from NAND outputs to 3-input NAND inputs and SR NAND inputs:
\draw (Nand4.in 1)--++(180:5mm)--++(90:3mm)--([yshift=-3mm] Nand3.out)--(Nand3.out);
\draw (Nand3.out) node[circ ](aux3){};

\draw (Nand3.in 2)--++(180:5mm)--++(-90:3mm)--([yshift=3mm] Nand4.out)--(Nand4.out);
\draw (Nand4.out) node[circ ](aux4){};

\draw (Nand4.out) node[circ, xshift = 5mm](aux1){};
\draw (aux1) --++(90:4cm)--++(-180:3cm) -| (J.in 1);

\draw (Nand3.out) node[circ, xshift = 8mm](aux2){};
\draw (aux2) --++(-90:4cm)--++(-180:3cm) -| (K.in 3);


\end{circuitikz}

%\caption{A gate-level schematic of one bit of memory, using a JK flip-flop circuit.}
%\end{center}
%\end{figure}

%\end{document}
