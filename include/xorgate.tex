\def\y{8}
\def\x{9}

\begin{circuitikz}

% First NOT gate, Input A
\draw 
	(0,\y+5) node[ocirc](ain) {} % IN node
	(0,\y+5) node[above] {{\color{red}$A$}} % A  label
	
%	(6.75,\y+4) node[short](aout){} % OUT node
	(7,\y+4) node[above] {{\color{red}$A'$}} % NOT A label
	(5,\y+4) |- (6.75,\y+4)
	(5,\y+4) node[circ](){}
	(3,\y+3) node[circ](){}

% Vdd:
	(5,\y+6) node[vcc](avin){}
    (4.5,\y+6) node[above] {$V_{in}$} % Vin

% GND
    (5,\y) node[ground](aground){}
%   (7.75,0.5) node[below] {{\color{red}$GND$}} % GND

% P-type FET
	(5,\y+5) node[pmos, emptycircle, xscale=1](Q1){}
	(5.5,\y+5) node[above]{$Q_1$}

% N-type FET
	(5,\y+3) node[nmos, rotate=0](Q2){}
	(5.5,\y+2.5) node[above]{$Q_2$}

% Resistors:
	(3,\y+3) to [R, l^=$R_1$](3,\y)
	(Q2.S) to [R, -*, l^=$R_2$](aground) 
	(ain) to [R, -*, l^=$R_3$](3,\y+5)
;

% nets:
\draw[cyan, thick](avin) |- (Q1.S);

\draw
	(Q1.D) -* (Q2.D)
	(3,\y) |- (aground)
;
\draw[cyan, thick]
    (3,\y+5) |- (Q2.G)
	(3,\y+5) |- (Q1.G)
;


% Second NOT gate, Input B
\draw 
	(0,5) node[ocirc](in) {} % IN node
	(0,5) node[above] {{\color{red}$B$}} % B label
	
	(6.5,4) node[circ](notb){} % OUT node
	(6.25,4) node[above] {{\color{red}$B'$}} % NOT B label
	(5,4) |- (notb)
	(5,4) node[circ](){}

% Vdd:
	(5,6.0) node[vcc](vin){}
    (4.5,6.0) node[above] {$V_{in}$} % Vin

% GND
    (5,0) node[ground](ground){}
%   (7.75,0.5) node[below] {{\color{red}$GND$}} % GND

% P-type FET
	(5,5) node[pmos, emptycircle, xscale=1](Q3){}
	(5.5,5.0) node[above]{$Q_3$}

% N-type FET
	(5,3) node[nmos, rotate=0](Q4){}
	(5.5,2.5) node[above]{$Q_4$}

% Resistors:
	(3,3) to [R, l^=$R_4$](3,0)
	(Q4.S) to [R, -*, l^=$R_5$](ground) 
	(in) to [R, -*, l^=$R_6$](3,5)

% nets:
	(vin) |- (Q3.S)
	(3,3) |- (3,5)
	(3,3) node[circ](){}	
	(Q3.D) |- (Q4.D)
	(3,0) |- (ground)
	(3,3) |- (Q4.G)
	(3,5) |- (Q3.G)
;


\draw
% P-type FET
	(\x,\y+5) node[pmos, emptycircle, xscale=1](Q5){}
	(\x+0.5,\y+5) node[above]{$Q_5$}

% P-type FET
	(\x,\y+3) node[pmos, emptycircle, xscale=1](Q6){}
	(\x+0.5,\y+3) node[above]{$Q_6$}

% P-type FET
	(\x+3,\y+5) node[pmos, emptycircle, xscale=1](Q7){}
	(\x+3.5,\y+5) node[above]{$Q_7$}

% P-type FET
	(\x+3,\y+3) node[pmos, emptycircle, xscale=1](Q8){}
	(\x+3.5,\y+3) node[above]{$Q_8$}

% Vdd:
	(\x,\y+6) node[vcc](cvin){}
    (\x+0.5,\y+6) node[above] {$V_{in}$} % Vin

% Vdd:
	(\x+3,\y+6) node[vcc](dvin){}
    (\x+3.5,\y+6) node[above] {$V_{in}$} % Vin

% Nets:
	(Q5.D) to (Q6.S)
	(Q7.D) to (Q8.S)
	(cvin) to (Q5.S)
	(dvin) to (Q7.S)
;


\draw
% N-type FET
	(\x,\y-1) node[nmos, xscale=1](Q9){}
	(\x+0.5,\y-1) node[above]{$Q_9$}

% N-type FET
	(\x,\y-3) node[nmos, xscale=1](Q10){}
	(\x+0.5,\y-3) node[above]{$Q_{10}$}

% N-type FET
	(\x+3,\y-1) node[nmos,  xscale=1](Q11){}
	(\x+3.5,\y-1) node[above]{$Q_{11}$}

% N-type FET
	(\x+3,\y-3) node[nmos, xscale=1](Q12){}
	(\x+3.5,\y-3) node[above]{$Q_{12}$}

% Grounds:
    (\x,2) node[ground](ground2){}
    (\x+5,-1) node[ground](ground3){}

% Current-limiting resistors:
	(Q10.S) to [R, -, l^=$R_7$](ground2) 
	(Q12.S) to [R, -, l^=$R_8$](12,2) 
	
% Nets:
	(Q11.S) to (Q12.D)
	(Q9.S) to (Q10.D)
	
	(Q8.D) to (Q11.D)
	(Q6.D) to (Q9.D)
	
	(\x, \y+4.0) -| (\x+3, \y+4.0)	
	(\x,\y+4.0) node[circ](){}
	(\x+3,\y+4.0) node[circ](){}

	(12,2) -- (14,2) -- (ground3)
;


% A signal:
\draw[thick, cyan]
	(3,\y+5) -- (3, \y+7) -- (6, \y+7) -- (6,\y-1) -- (Q9.G)
	(6,\y+5) node[circ, color=cyan](){}
	(6,\y+5) -| (Q5.G)
;


% B signal:
\draw[thick, blue]
	(3,4) -| (2, 4) -- (2, -1) -| (7,-1) -- (7,5) --  (7.5,5) -- (Q10.G)
	(3,4) node[circ, color=blue](){}
	(7.5,5) node[circ, color=blue](){}
	(7.5,5) -- (7.5,10) --  (10,10) -- (10,13) -- (Q7.G)
;


% NOT A signal:
\draw[thick, orange]
	(6.75,\y+4) -- (7.25, 12) -- (7.25, -1.0) -- (10.5, -1.0) -- (10.5,7) -- (Q11.G)
	(7.25,11) -- (Q6.G)
	(7.25,11) node[circ, color=orange](){}
	(6.75,\y+4) -- (5,\y+4)
;


% NOT B signal, a white line with a very thick black border:
\draw[line width=3, color=black]
	(5,4) -- (6.5, 4) -| (6.5, 0.5) -- (10,0.5) -- (10,5) -- (Q12.G)
	(10,3) -- (10,8.0) -- (11, 8.0) -- (Q8.G)
;


\draw[thick, color=white]
	(5,4) -- (6.5,4) --  (6.5,0.5) -- (10,0.5) -- (10,5) -- (Q12.G)
	(10,3) -- (10,8.0) -- (11, 8.0) -- (Q8.G)
;


\draw
% Pull-down resistors, R9 (NOT_A) and R10 (NOT_B)
	(10.5,-1.0) node[circ, color=orange](){}
	(10.5,-1.0) to [R, -* , l^=$R_9$](14,-1)

	(10.0,0.5) node[circ, color=black](){}
	(10.0,0.5) to [R, -* , l^=$R_{10}$](14,0.5)
;

% XOR signal:

\draw[color=brown, thick]
	(\x, \y+1) -* (\x+3, \y+1)
	(\x,\y+1) node[circ](){}
	(\x+3,\y+1) node[circ](){}
	
	(\x+5,\y+1) node[circ](xor){} % OUT node
	(\x+4.25,\y+1) node[above] {{\color{red}$A~XOR~B$}} % XOR label
	(\x+3,\y+1) |- (xor)
;


% Labeling and explanation
\draw
 (15,6) node[align=left, text width = 3cm](exp1){This pair of AND gates grounds the circuit if both inputs are the same.}

 (15,12) node[align=left, text width = 4cm](exp1){This pair of AND gates controls circuit power input. Each (linked) pair only functions if the \\ inputs are different.}

 (1,7) node[align=left, text width = 2.75cm](exp1){Two NOT gates enable comparisons among inputs.}
;

\end{circuitikz}
