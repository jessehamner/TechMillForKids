\documentclass[12pt]{article}
\usepackage[left=1in,right=1in,top=1in,bottom=1in]{geometry} 
\usepackage{xcolor}
\usepackage{tikz}
\usepackage[american, EFvoltages, cuteinductors]{circuitikz}

\definecolor{rltgreen}{rgb}{0,0.6,0}
\definecolor{rltred}{rgb}{0.8,0,0}

\begin{document}
\thispagestyle{empty}

% In this function,
% y coord is arg 1
% bit pattern name (zero, e.g.) is arg 2
% numeric indicator (0, e.g.) is arg 3

\newcommand{\adderblock}[3]{

	\begin{scope}

	  \draw (0,#1)
	    [xshift = 2.5cm]
	    node[one bit adder, scale=0.75](fa#2){}
	    node[xshift=0.75cm, yshift=1cm](fa#2carrylabel){\footnotesize $Carry_#3$}
  	    node[ocirc, xshift=1.5cm, label={[label distance=0mm]0:$Sum_#3$}](fa#2sum){}
	  ;

	  
  
    \draw (0.5,#1)
 	  [yshift=-0.65cm] 
	  node[american xor port](axorb#2){}
	  node[ocirc, xshift = -2.2cm, yshift = -0.28cm](axorb#2node){}
   	  node[xshift = -2.6cm, yshift = 0mm](){{\color{red}$A_{#3}$}}
	  node[xshift = 0.2cm, yshift = 6mm] {{\footnotesize{$XOR_{#3}$}}}
   	  node[ocirc, xshift = -4.5cm, yshift = 0.32cm](m#2node){}
	  node[xshift = -4.0cm, yshift = 0.6cm, color=rltred]{\color{rltred}$M_{#3}$}
	;
  

  
    \draw (0,#1)
	  [yshift=0.65cm]
	  node[ocirc, xshift = -2.0cm](b#2node){}
   	  node[xshift = -2.3cm, yshift = 3mm](){{\color{blue}$B_{#3}$}}
	  ;
  
	
  
    \draw (0,#1)
	  [color=blue] 
	  (b#2node) -- (fa#2.lpin 1) 
	;
  

    \draw
	  [color=rltred] 
	  (m#2node) -- (axorb#2.in 1) 
	;


  \draw (0,#1)
	  (axorb#2node) -- (axorb#2.in 2)
	  (fa#2.rpin 1) -- (fa#2sum)
	;
	
  \draw(0, #1)[color=brown, very thick]
    (axorb#2.out) -- (fa#2.lpin 2)
    
    ;
  \end{scope}
}

\begin{figure}[h!]
\begin{center}

\begin{circuitikz}

\tikzset{one bit adder/.style={muxdemux, 
  muxdemux def={Lh=4, NL=2, Rh=2, NR=1, NB=1, NT=1, w=1.5,
  inset w=0.5, inset Lh=2, inset Rh=1.5}}
}

% Inputs and NOT gates: (for illustrative purposes; in reality, add/subtract are switched
% via a simple high-low signal line and complement is handled with XOR gates, with carry-in
% handling the "add one to get the two's complement" into the LSB full adder)
\begin{scope}

	\adderblock{1.25}{zero}{0}
		
	\adderblock{4.25}{one}{1}

	\adderblock{7.25}{two}{2}
	
	\adderblock{10.25}{three}{3}

\end{scope}

% the add-one input to the least-significant-bit full adder, for two's complement:
\draw
	(2cm,-1.0cm) node[ocirc](carrynode) {} 
	node[below] {{\color{black}$Carry-In$}}
;

\draw
[color=rltgreen, style=thick]
    (carrynode) -| (fazero.bpin 1)
    (fazero.tpin 1) -| (faone.bpin 1)
    (faone.tpin 1) -| (fatwo.bpin 1)
    (fatwo.tpin 1) -| (fathree.bpin 1)
    (fathree.tpin 1) to (2.5,13) node[ocirc, label={[label distance=0mm]75:{{\color{black}$Overflow$}}}](overflow){}
;

\draw(-4cm, 13cm)
[color=rltred]
node[ocirc, color=rltred](dosubtractionnode){}
node[xshift=1.5cm, yshift = 0.2cm](dosubtractionlabel){\color{rltred}\emph{Do Subtraction}}
;

\draw
[color=rltred, style=thick]
  (carrynode) -| (-4cm, 0) -| (mzeronode) -| (monenode)  -| (mtwonode)  -- (mthreenode) -- (dosubtractionnode)
;



\end{circuitikz}

\caption{How addition and subtraction gets implemented with logic gates to either subtract {\color{red}$A$}, a four digit binary number (something like \texttt{0010}, or ``2"), from {\color{blue}$B$} (perhaps \texttt{1000}, or ``8''), or add the two numbers instead. Each pair of bits goes into a full adder (the trapezoid-shaped symbol). To use subtraction, the M-line is held to a positive voltage, effectively turning the $XOR$ gate into a $NOT$ gate.}
\end{center}
\end{figure}

\end{document}
