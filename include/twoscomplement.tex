%\documentclass[12pt]{article}
%\usepackage[left=1in,right=1in,top=1in,bottom=1in]{geometry} 
%\usepackage{xcolor}
%\usepackage{tikz}
%\usepackage[american, EFvoltages, cuteinductors]{circuitikz}

%\definecolor{rltgreen}{rgb}{0,0.6,0}

%\begin{document}
%\thispagestyle{empty}

% In this function,
% y coord is arg 1
% bit pattern name (zero, e.g.) is arg 2
% numeric indicator (0, e.g.) is arg 3

\begin{figure}[h!]
\begin{center}

\begin{circuitikz}

\tikzset{one bit adder/.style={muxdemux, 
  muxdemux def={Lh=4, NL=2, Rh=2, NR=1, NB=1, NT=1, w=1.5,
  inset w=0.5, inset Lh=2, inset Rh=1.5}}
}

% Inputs and NOT gates: (for illustrative purposes; in reality, add/subtract are switched
% via a simple high-low signal line and complement is handled with XOR gates, with carry-in
% handling the "add one to get the two's complement" into the LSB full adder)
\begin{scope}

	\adderblock{1.25}{zero}{0}
		
	\adderblock{4.25}{one}{1}

	\adderblock{7.25}{two}{2}
	
	\adderblock{10.25}{three}{3}

\end{scope}

% the add-one input to the least-significant-bit full adder, for two's complement:
\draw
	(-1cm,-1.0cm) node[ocirc](carrynode) {} 
	node[left] {{\color{black}$Carry-In$}}
;

\draw
[color=rltgreen, style=thick]
    (carrynode) -| (fazero.bpin 1)
    (fazero.tpin 1) -| (faone.bpin 1)
    (faone.tpin 1) -| (fatwo.bpin 1)
    (fatwo.tpin 1) -| (fathree.bpin 1)
    (fathree.tpin 1) to (2.5,13) node[ocirc, label={[label distance=0mm]75:{{\color{black}$Overflow$}}}](overflow){}
;


\end{circuitikz}

\caption{How subtraction gets implemented with logic gates to subtract {\color{red}$A$}, a four digit binary number (something like \texttt{0010}, or ``2"), from {\color{blue}$B$} (perhaps \texttt{1000}, or ``8''). Each pair of bits goes into a full adder (the trapezoid-shaped symbol). Each of these pieces is explained in this workshop -- you \emph{really can} understand this by the end of the day!}
\end{center}
\end{figure}

%\end{document}
