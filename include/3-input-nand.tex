%\documentclass[12pt]{article}
%\usepackage[left=1in,right=1in,top=1in,bottom=1in]{geometry} 
%\usepackage{xcolor}
%\usepackage{tikz}
%\usepackage[american, EFvoltages, cuteinductors]{circuitikz}
%\begin{document}
%\thispagestyle{empty}

\begin{figure}[h!]
\begin{center}

\begin{circuitikz}


% Inputs and 3 NAND gates:
\begin{scope}
	\draw
		(0,1.0cm) 
		node[american nand port, anchor=in 2, number inputs = 2](N1){}
		node[ocirc, xshift = -0.5cm, yshift = 0.55cm](anode){}
	    node[xshift = -0.9cm, yshift = 7mm](){{\color{red}$A$}}
	    
        node[ocirc, xshift = -0.5cm, yshift = 0cm](bnode){}	
        node[xshift = -0.9cm, yshift = 0.0mm](){{\color{red}$B$}}
		
		node[ocirc, xshift = -0.5cm, yshift = -0.6cm](cnode){}	
        node[xshift = -0.9cm, yshift = -6mm](){{\color{red}$C$}}
		
		node[xshift = 0.6cm, yshift = 1.2cm] {{\footnotesize{$NAND_1$}}} 
        
        node[circ, xshift = 1.5cm, yshift = 0.3cm](N2tie){}
        
		(anode) |- (N1.in 1)
		(bnode) |- (N1.in 2)
		(N1.out) |- (N2tie)

	;

	\draw
		(2.0cm,1.0cm) 
		node[american nand port, anchor=in 2, number inputs = 2](N2){}
		
		node[xshift = 0.7cm, yshift = 1.2cm](){{\footnotesize{$NAND_2$}}}

		(N1.out) |- (N2.in 2)
		(N1.out) |- (N2.in 1)
	;


	\draw
		(4.0cm, 0.4cm) 
		node[american nand port, anchor=in 2, number inputs = 2](N3){}
		
		node[xshift = 0.8cm, yshift = 1.2cm](){{\footnotesize{$NAND_3$}}} 

        node[ocirc, xshift = 2.2cm, yshift = 0.275cm](nand3output){}

		(N2.out) |- (N3.in 1)
		(N3.out) |- (nand3output)
		node[xshift = 0.4cm, yshift = 0.5cm](){{\color{red}$Output$}}
		(cnode) -- (N3.in 2)
	;


	
\end{scope}


\end{circuitikz}

\caption{A three-input NAND gate created only from two-input NAND gates. The simplest version of a three-input NAND uses a single AND gate and a single NAND gate, but this example shows the potential to use NAND gates as universal gates, just like the NOR gates, above. NAND 2 is configured as a NOT gate, making NAND 1 and NAND 2 equal to an AND gate.}
\end{center}
\end{figure}

%\end{document}
