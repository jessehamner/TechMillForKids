\begin{figure}[ht!]
\begin{center}

\begin{tabular}{m{3.2in} m{2.5in}}
\hline\\[\negsep]


\begin{circuitikz}


% Gates:
\draw
	(4,2) node[xor port, fill=gray!40](xorGate) {} % xor gate
	(4.2,3.0) node[left] {$XOR$} % XOR label
	
	(4,0) node[and port, fill=gray!40](andGate) {} % AND gate
	(4.2,-1.0) node[left] {$AND$} % AND label
;

% Input nodes:
\draw
	(0,2.27) node[ocirc](ainput) {} % A node
	(0,2.3) node[left] {{\color{red}$A = 0$}} % A label

	(0,1.72) node[ocirc](binput) {} % B node
	(0,1.72) node[left] {{\color{red}$B = 0$}} % B label
;

% Output nodes:
\draw
	(5,2) node[ocirc](snode) {} % S node
;

\draw (xorGate.out) to (snode) % S output wiring
;

\draw (5,2) node[right] {{\color{red}$Sum$}} % S label
;

\draw
	(5,0) node[ocirc](cnode) {} % C node
	(andGate.out) to (cnode) % C output wiring
	(5,0) node[right] {{\color{red}$Carry$}} % C label
;

% Nets:
\draw (ainput) -- (xorGate.in 1) % xor A wiring
;

\draw (binput) -- (xorGate.in 2) % xor B wiring	
;

	
\draw (1.5,1.72) |- (andGate.in 1)
;

\draw (1,2.27) |- (andGate.in 2)
;


\draw (1,2.27) node[circ] {}
;

\draw (1.5,1.72) node[circ] {}
;



\end{circuitikz}
 & 
Neither input is 1. The XOR gate does not output high because both inputs are equal. The AND gate does not output high because both inputs are not 1. Adding 0 to 0 equals 0, and so the output is: $sum = 0 \times 1$, $carry = 0 \times 2$.\\[\sep]
\hline\\[\negsep]

\input{./include/halfadder1_example01.tex} &  
One input is 1, and one input is 0. The XOR gate outputs high because only one input is supplying voltage. The AND gate does not output high because both inputs are not 1. Adding 1 to 0 equals 1, and so the output is: $sum  = 1 \times 1$, $carry = 0 \times 2$.\\[\sep]
\hline\\[\negsep]

\input{./include/halfadder1_example02.tex} &  
Both inputs are 1. The XOR gate outputs low (ground, or zero) because both inputs are equal. The AND gate outputs high because both inputs are 1. Adding 1 to 1 equals 2, and so the output is: $sum  = 0 \times 1$, $carry = 1 \times 2$.\\[\sep]
\hline
\end{tabular}

\caption{The three possible conditions for a half adder. This series is a graphical representation of the truth table for a half adder.}
\label{fig:threehalfadders}
\end{center}
\end{figure}
