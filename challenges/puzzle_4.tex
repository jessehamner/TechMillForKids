\subsection*{Puzzle 4}

Say you had three inputs $A$, $B$, and $C$, and you want to know when a majority of them are ON or TRUE.
The question is simple enough: with three inputs, use connected logic gates in a way that can tell when the majority of inputs (here, that's at least two of the three) have a 1 as their input value.

\subsubsection*{Setting Up}

Start by taking a minute to think of the way to test if some pair of inputs are both 1. We will need to test each possible pair of inputs. That may seem like a big deal, but we can break it down into small chunks.

There's a way to represent all of the possible conditions in a compact way.
Let's start by saying a line over the letter indicates ``NOT true'', and no line over the letter indicates ``true''.
\begin{itemize}
\item If all of the inputs are off, then the condition can be represented by: $\overline{ABC}$. 
\item When all inputs are on, the shorthand would be $ABC$. 
\item When only A is off, the condition is represented by $\overline{A}BC$. 
\end{itemize}

We must first find any combination of the three signals that has two or more of them reporting ``yes''. Then we can determine how we might test the signals. How might you break up the problem into tests of two inputs? 

Returning to our setup: how many combinations are there in total?

\begin{enumerate}
\item Each input can be 1 or 0 -- there are exactly two possible states
\item There are three inputs in total, and each can be on or off
\item Each input is not affected by any other input (the inputs are not tied together in any way)
\end{enumerate}

We need a truth table!
The truth table will have $2 \times 2 \times 2$ (that is, $2^3$) possible combinations. 
So let's make a table by counting from 0 to 7 in binary.
We will assign one of the A, B, or C inputs to each column ($2^0$, $2^1$, and $2^2$).

\renewcommand{\arraystretch}{1.2}

\begin{tabular}{m{2in} m{3in}}

\begin{tabular}{cccc}
\hline
A & B & C & Symbol\\
\hline
0 & 0 & 0 & $\overline{ABC}$\\
0 & 0 & 1 & $\overline{AB}C$\\
0 & 1 & 0 & $\overline{A}B\overline{C}$\\
0 & 1 & 1 & $\overline{A}BC$\\
1 & 0 & 0 & $A\overline{BC}$\\
1 & 0 & 1 & $A\overline{B}C$\\
1 & 1 & 0 & $AB\overline{C}$\\
1 & 1 & 1 & $ABC$\\
\hline
\end{tabular}

&

To cover every possible case, half of of each column will be ``true'' and half will be ``false''.
For the left-most column (``A''), the first four column values are 0 and the next four are 1.
The right-most column alternates ``on-off'' four times, and the middle column alternates off and on every other row.
It looks like we are counting in binary from 0 to 7, because we are! That way we will know we have covered every possible condition that we have to test.

\\
\end{tabular}

\renewcommand{\arraystretch}{1.0}

The table shows us all possible conditions for these three inputs. Now add a results column, showing the answer our logic circuit should produce. Remember, if any two inputs in the row are true, then the result column will be true. Otherwise, the result is false.

\begin{tabular}{m{2in} m{3in}}
\renewcommand{\arraystretch}{1.2}
\begin{tabular}{ccccc}
\hline
A & B & C & Symbol           & Result\\
\hline
0 & 0 & 0 & $\overline{ABC}$ & 0\\
0 & 0 & 1 & $\overline{AB}C$ & 0\\
0 & 1 & 0 & $\overline{A}B\overline{C}$ & 0\\
0 & 1 & 1 & $\overline{A}BC$ & 1\\
1 & 0 & 0 & $A\overline{BC}$ & 0\\
1 & 0 & 1 & $A\overline{B}C$ & 1\\
1 & 1 & 0 & $AB\overline{C}$ & 1\\
1 & 1 & 1 & $ABC$ & 1\\
\hline
\end{tabular}

&

The fourth, sixth, seventh, and eighth rows all have two or more inputs set to ``on''. The results (``R'') column is 1/``true'' for these four rows, and 0/``false'' for the rest. 

\\
\end{tabular}
\renewcommand{\arraystretch}{1.0}

The logic can be expressed like this: 

($A$ AND $B$) OR ($A$ AND $C$) OR ($B$ AND $C$) OR ($A$ AND $B$ AND $C$).

So if any of these individual conditions is true, the result is true.
Note that all three values do not have to be true for the result to be true. Any two true inputs gives the same result as all three inputs being on. So there is no need to specifically test for all three inputs being true.

