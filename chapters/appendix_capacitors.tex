\section{A Bit More About Capacitors}

The Leyden jar was one of the first capacitors invented: metal foil was placed inside a glass jar, and wrapped around the outside of a glass jar, but neither foil gets near the top of the jar. The glass barrier between the foil sheets (inside and outside) allows a charge to build up between the foil without allowing the electrical charge to move through the glass. Leyden jars didn't hold much charge, but the concept it demonstrated has not changed.

Capacitors are usually made with two metal plates that are on top of each other (or wrapped around each other), but that do not actually touch. When powered, they allow energy to be stored inside an electrical field. Because the plates need a lot of area to store even a small amount of charge, the plates are usually rolled up into some other shape, such as a cylinder. Sometimes, other shapes of capacitors are used for special purposes. 

\bigskip
\stbox{6.0in}{
\emph{Experiment:} make a variable capacitor from aluminum foil, paper, and a paper towel tube. The paper should go around the roll only once. Cut the foil one-quarter or one-half inch smaller than the paper on all sides. Tape a small wire to one corner of the foil, if possible using metallic/conductive tape\footnote{While it is possible to solder to aluminum foil, really there are much better and safer ways to attach a wire to the foil for the purposes of this workshop.}. Tape the foil to the paper. Tape one edge of the paper, foil side down, against the cardboard tube. Make another paper/foil combination. Wrap the paper/foil around the cardboard tube, but tape the paper only to itself, and just loose enough to slide a little. 

Another example, using a coke can, can be found {\color{webblue}\href{http://www.tompolk.com/crystalradios/cokecancapacitor.html}{here}}.

\bigskip

It is also possible to make a regular capacitor with the paper-foil layers, separated by flat sheets of cardboard. Keep track of the ``up'' and ``down'' sides! Note that it is pretty easy to gang each ``plate'' together and make capacitor with a larger value; tying all the anodes together, and all the cathodes together, makes a larger capacitor.
}
