\section{Computer Math}

In this section, I introduce a few vocabulary terms that help you understand how computers do certain things, like add two (binary) numbers. In most respects, adding these numbers is no different from how you already add numbers. When the numbers in any column is too big to be held in that column, you must ``carry'' the extra number to the next larger column. The difference, and it is slightly awkward at first, is remembering that any column can only hold 0 or 1 -- if you reach 2, you have to carry the ``two'' to the next column.

\subsection*{Addition}

Adding two values is straightforward. Simply align the values on the least significant bit and add each column, moving any ``carry'' to the bit one position left, just like you already do with ``regular'' (base-10) arithmetic.

\begin{verbatim}
  Binary:     Decimal:

  0001 0110   (16 + 0 + 4 + 2 + 0) =   22
+ 0000 0011    (0 + 0 + 0 + 1 + 2) =    3
===========                          ====
  0001 1001   (16 + 8 + 0 + 0 + 1) =   25
\end{verbatim}


\bigskip

\noindent Your turn! Add these numbers in binary, column by column, carrying each ``2'' as needed. After you add them, convert each number to decimal to check your work.

\bigskip

\begin{tabular}{p{3in} | c  p{3in} }
\hline
\\
\begin{minipage}{2.95in}
\begin{verbatim}
   Binary:       Decimal:

  0000 0000
+ 0000 0011
===========      ====

___________      ____
\end{verbatim}
\end{minipage}

&&

\begin{minipage}{2.95in}
\begin{verbatim}
   Binary:       Decimal:

  0000 0010
+ 0000 0010
===========      ====

___________      ____
\end{verbatim}
\end{minipage}

\\
\hline
\\

\begin{minipage}{2.95in}
\begin{verbatim}
   Binary:       Decimal:

  0000 0110
+ 0000 0011
===========      ====

___________      ____
\end{verbatim}
\end{minipage}

&&

\begin{minipage}{2.95in}
\begin{verbatim}
   Binary:       Decimal:

  0000 1010
+ 0001 1011
===========      ====

___________      ____
\end{verbatim}
\end{minipage}

\\
\hline
\end{tabular}

There is more information about binary arithmetic in the Appendix -- you can see how to represent negative numbers, and how to subtract numbers, in binary, just the way a computer does. But for now, you know enough to move on.

\vfill

%\documentclass[12pt]{article}
%\usepackage[left=1in,right=1in,top=1in,bottom=1in]{geometry} 
%\usepackage{xcolor}
%\usepackage{tikz}
%\usepackage[american, EFvoltages, cuteinductors]{circuitikz}

%\definecolor{rltgreen}{rgb}{0,0.6,0}

%\begin{document}
%\thispagestyle{empty}

% In this function,
% y coord is arg 1
% bit pattern name (zero, e.g.) is arg 2
% numeric indicator (0, e.g.) is arg 3

\begin{figure}[h!]
\begin{center}

\begin{circuitikz}

\tikzset{one bit adder/.style={muxdemux, 
  muxdemux def={Lh=4, NL=2, Rh=2, NR=1, NB=1, NT=1, w=1.5,
  inset w=0.5, inset Lh=2, inset Rh=1.5}}
}

% Inputs and NOT gates: (for illustrative purposes; in reality, add/subtract are switched
% via a simple high-low signal line and complement is handled with XOR gates, with carry-in
% handling the "add one to get the two's complement" into the LSB full adder)
\begin{scope}

	\adderblock{1.25}{zero}{0}
		
	\adderblock{4.25}{one}{1}

	\adderblock{7.25}{two}{2}
	
	\adderblock{10.25}{three}{3}

\end{scope}

% the add-one input to the least-significant-bit full adder, for two's complement:
\draw
	(-1cm,-1.0cm) node[ocirc](carrynode) {} 
	node[left] {{\color{black}$Carry-In$}}
;

\draw
[color=rltgreen, style=thick]
    (carrynode) -| (fazero.bpin 1)
    (fazero.tpin 1) -| (faone.bpin 1)
    (faone.tpin 1) -| (fatwo.bpin 1)
    (fatwo.tpin 1) -| (fathree.bpin 1)
    (fathree.tpin 1) to (2.5,13) node[ocirc, label={[label distance=0mm]75:{{\color{black}$Overflow$}}}](overflow){}
;


\end{circuitikz}

\caption{How subtraction gets implemented with logic gates to subtract {\color{red}$A$}, a four digit binary number (something like \texttt{0010}, or ``2"), from {\color{blue}$B$} (perhaps \texttt{1000}, or ``8''). Each pair of bits goes into a full adder (the trapezoid-shaped symbol). Each of these pieces is explained in this workshop -- you \emph{really can} understand this by the end of the day!}
\end{center}
\end{figure}

%\end{document}

