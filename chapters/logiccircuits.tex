\section{Logic Circuits}

\subsection*{A Half Adder}

A \emph{half adder} demonstrates the way in which computer logic gates can correctly add (binary) numbers, though here it can add only two \emph{binary digits} (``bits''). Thus, each half adder can only count to \emph{two}, but that is enough!  Long cascades of half adders and \emph{full adders} enable computers to count large numbers. Below, see two examples of half adders. The second example replaces the XOR gate with a series of simpler gates, but the results are the same. Each half adder takes two bits and adds them, passing on the bits for further work or as an answer itself.

\bigskip

\begin{figure}[!ht]
\begin{center}
\begin{tabular}{m{3.25in} m{2.0in}}



\begin{circuitikz}

\draw

% Gates:
	(4,2) node[xor port](xorGate) {} % xor gate
	(4.2,3.0) node[left] {$XOR$} % XOR label
	
	(4,0) node[and port](andGate) {} % AND gate
	(4.2,-1.0) node[left] {$AND$} % AND label

% Input nodes:
	(0,2.27) node[ocirc](ainput) {} % A node
	(0,2.3) node[left] {{\color{red}$A$}} % A label

	(0,1.72) node[ocirc](binput) {} % B node
	(0,1.72) node[left] {{\color{red}$B$}} % B label

% Output nodes:
	(5,2) node[ocirc](snode) {} % S node
	(xorGate.out) to (snode) % S output wiring
	(5,2) node[right] {{\color{red}$Sum$}} % S label

	(5,0) node[ocirc](cnode) {} % C node
	(andGate.out) to (cnode) % C output wiring
	(5,0) node[right] {{\color{red}$Carry$}} % C label
;

% Nets:
\draw
	(ainput) |- (xorGate.in 1) % xor A wiring
	(binput) |- (xorGate.in 2) % xor B wiring	
;
\draw[blue, thick]
	(1,2.27) node[circ] {}
	(1.5,1.72) node[circ] {}
	
	(1.5,1.72) |- (andGate.in 1)
	(1,2.27) |- (andGate.in 2)
;

\end{circuitikz}


&

\begin{tabular}{ll | cc | c}
\multicolumn{5}{c}{\textbf{Truth Table}}\\
\hline\\[\negsep]
\textbf{A} & \textbf{B} & \textbf{Sum} & \textbf{Carry} & \textbf{Decimal}\\
\hline
0 & 0 & 0 & 0 & 0 \\
1 & 0 & 1  & 0 & 1 \\
0 & 1 & 1  & 0 & 1 \\
1 & 1 & 0  & 1 & 2 \\
\hline
\end{tabular}

\\

\end{tabular}

\caption{A half-adder circuit. It adds two bits together. If the result is larger than 1, the ``carry" signal is high.}
\end{center}
\end{figure}

Figure \ref{fig:threehalfadders} shows each of the three possible outcomes of a half adder. Make sure you understand how each result happens.

\begin{figure}[!hb]
\begin{center}

\begin{circuitikz}

% Inputs:
\draw
	(0,5.5) node[ocirc](anode) {} %A node
	(0.25,6.0) node[left] {{\color{red}$A$}} %A label
	
	(0.75,5.5) node[ocirc](bnode) {} %B node
	(1.1,6.0) node[left] {{\color{red}$B$}} %B label
;

% NOT gates:
\draw
	(2,4.25) node[american not port](nota){}   % not gate #1 input
	(2,1.75) node[american not port](notb){}   % not gate #2 input
;

% AND gates:
\draw
	(4.5,3.75) node[american and port](andGate1) {} % AND gate #1 
	(4.5,2.25) node[american and port](andGate2) {} % AND gate #2 
	(5,0) node[american and port](andGate3) {} % AND gate #3
;

% OR gate:
\draw
	(7.0,3) node[american or port](orGate) {}
;

% From the NOT gates out to the AND gates and OR gate:
\draw[red!40!blue, thick]
	(nota.out) to[short](andGate1.in 1) % NOT output net goes into andGate1.in 1
;

\draw[orange, thick]	
	(notb.out) to[short](andGate2.in 2) % NOT output net goes into andGate2.in 2
;


\draw[thick]
	(0,3.5) node[circ](atoand1){}
	(atoand1) to [short](andGate1.in 2)
;

\draw[green!50!black, thick]
	(andGate1.out) to [short](5.0,3.75)
	(5.0,3.75)  |- (orGate.in 1)
;

\draw[brown!60!black, thick]
	(andGate2.out) to [short](5.0,2.25)
	(5.0,2.25)  |- (orGate.in 2)
;

\draw[blue, thick]
	(0.75,4.25) node[circ, color=blue](btonot1){}	
	(bnode) |- (nota.in)
;

\draw[thick]
	(0,5.5) |- (notb.in)
	(0,1.75) node[circ](atonot2){}
	(atonot2) |- (andGate3.in 2)
;

% B input line nets:
\draw[blue, thick]
	(0.75,4.25) to [short](0.75,2.5)
	(0.75,2.52) node[circ, color=blue]{}
	(0.75,2.52) |- (andGate2.in 1)
	(0.75,2.5) |- (andGate3.in 1)
;

\draw	
% Output nodes:
	(8,3) node[ocirc](snode) {} %S node
	(8.25,3) node[right] {{\color{red}$Sum$}} % S label

	(5.75,0) node[ocirc](cnode) {} % C node
	(6.0,0) node[right] {{\color{red}$Carry$}} % C label

	(andGate3.out) to[short](cnode) % C output
	(orGate.out) to [short](snode) % S output 
	
% Labeling:
	(1.75,4.85) node[right] {{\footnotesize{$NOT_1$}}} 
	(1.75,1.15) node[right] {{\footnotesize{$NOT_2$}}} 
	(3.5,4.6) node[right] {{\footnotesize{$AND_1$}}} 
	(3.5,1.35) node[right] {{\footnotesize{$AND_2$}}}
	(4.5,-0.6) node[right] {{\footnotesize{$AND_3$}}}
	(6.5,3.7) node[right] {{\footnotesize{$OR$}}}
    
    ;

\end{circuitikz}


\caption{Another half-adder circuit. The exclusive-OR gate has been replaced by an equivalent circuit made up of five simpler gates. You can see why XOR requires so many transistors!}
\end{center}
\end{figure}

\begin{figure}[ht!]
\begin{center}

\begin{tabular}{m{3.2in} m{2.5in}}
\hline\\[\negsep]


\begin{circuitikz}


% Gates:
\draw
	(4,2) node[xor port, fill=gray!40](xorGate) {} % xor gate
	(4.2,3.0) node[left] {$XOR$} % XOR label
	
	(4,0) node[and port, fill=gray!40](andGate) {} % AND gate
	(4.2,-1.0) node[left] {$AND$} % AND label
;

% Input nodes:
\draw
	(0,2.27) node[ocirc](ainput) {} % A node
	(0,2.3) node[left] {{\color{red}$A = 0$}} % A label

	(0,1.72) node[ocirc](binput) {} % B node
	(0,1.72) node[left] {{\color{red}$B = 0$}} % B label
;

% Output nodes:
\draw
	(5,2) node[ocirc](snode) {} % S node
;

\draw (xorGate.out) to (snode) % S output wiring
;

\draw (5,2) node[right] {{\color{red}$Sum$}} % S label
;

\draw
	(5,0) node[ocirc](cnode) {} % C node
	(andGate.out) to (cnode) % C output wiring
	(5,0) node[right] {{\color{red}$Carry$}} % C label
;

% Nets:
\draw (ainput) -- (xorGate.in 1) % xor A wiring
;

\draw (binput) -- (xorGate.in 2) % xor B wiring	
;

	
\draw (1.5,1.72) |- (andGate.in 1)
;

\draw (1,2.27) |- (andGate.in 2)
;


\draw (1,2.27) node[circ] {}
;

\draw (1.5,1.72) node[circ] {}
;



\end{circuitikz}
 & 
Neither input is 1. The XOR gate does not output high because both inputs are equal. The AND gate does not output high because both inputs are not 1. Adding 0 to 0 equals 0, and so the output (sum $ = 0 \times 1$, carry $ = 0 \times 2$) is correct.\\[\sep]
\hline\\[\negsep]

\begin{circuitikz}

% Gates:
\draw
	(4,2) node[xor port, fill=yellow](xorGate) {} % xor gate
	(4.2,3.0) node[left] {$XOR$} % XOR label
	
	(4,0) node[and port](andGate) {} % AND gate
	(4.2,-1.0) node[left] {$AND$} % AND label
;

% Input nodes:
\draw
	(0,2.27) node[ocirc](ainput) {} % A node
	(0,2.3) node[left] {{\color{red}$A = 1$}} % A label

	(0,1.72) node[ocirc](binput) {} % B node
	(0,1.72) node[left] {{\color{red}$B = 0$}} % B label
;

% Output nodes:
\draw
	(5,2) node[ocirc, fill=red](snode) {} % S node
;

\draw[red, very thick](xorGate.out) to (snode) % S output wiring
;

\draw
	(5,2) node[right] {{\color{red}$Sum$}} % S label
;
\draw
	(5,0) node[ocirc](cnode) {} % C node
	(andGate.out) to (cnode) % C output wiring
	(5,0) node[right] {{\color{red}$Carry$}} % C label
;

% Nets:
\draw[red, very thick](ainput) -- (xorGate.in 1) % xor A wiring
;

\draw (binput) -- (xorGate.in 2) % xor B wiring	
;

	
\draw[black] (1.5,1.72) |- (andGate.in 1)
;

\draw[red, very thick] (1,2.27) |- (andGate.in 2)
;


\draw[black] (1,2.27) node[circ] {}
;

\draw[black] (1.5,1.72) node[circ] {}
;


\end{circuitikz}
 &  
One input is 1, and one input is 0. The XOR gate outputs high because only one input is supplying voltage. The AND gate does not output high because both inputs are not 1. Adding 1 to 0 equals 1, and so the output (sum $ = 1 \times 1$, carry $ = 0 \times 2$) is correct.\\[\sep]
\hline\\[\negsep]



\begin{circuitikz}


% Gates:
\draw
	(4,2) node[xor port, fill=gray!40](xorGate) {} % xor gate
	(4.2,3.0) node[left] {$XOR$} % XOR label
;

\draw
	(4,0) node[and port, fill=yellow](andGate) {} % AND gate
	(4.2,-1.0) node[left] {$AND$} % AND label
;

% Input nodes:
\draw
	(0,2.27) node[ocirc](ainput) {} % A node
	(0,2.3) node[left] {{\color{red}$A = 1$}} % A label

	(0,1.72) node[ocirc](binput) {} % B node
	(0,1.72) node[left] {{\color{red}$B = 1$}} % B label
;

% Output nodes:
\draw (5,2) node[ocirc](snode) {} % S node
;

\draw[black](xorGate.out) to (snode) % S output wiring
;

\draw (5,2) node[right] {{\color{red}$Sum$}} % S label
;

\draw (5,0) node[ocirc, fill=red](cnode) {} % C node
;

\draw[red, very thick](andGate.out) to (cnode) % C output wiring
;

\draw (5,0) node[right] {{\color{red}$Carry$}} % C label
;

% Nets:
\draw[red, very thick](ainput) -- (xorGate.in 1) % xor A wiring
;

\draw[red, very thick] (binput) -- (xorGate.in 2) % xor B wiring	
;

	
\draw[red, very thick] (1.5,1.72) |- (andGate.in 1)
;

\draw[red, very thick] (1,2.27) |- (andGate.in 2)
;


\draw[black] (1,2.27) node[circ] {}
;

\draw[black] (1.5,1.72) node[circ] {}
;



\end{circuitikz}
 &  
Both inputs are 1. The XOR gate outputs low (ground, or zero) because both inputs are equal. The AND gate outputs high because both inputs are 1. Adding 0 to 2 equals 2, and so the output (sum $ = 0 \times 1$, carry $ = 1 \times 2$) is correct.\\[\sep]
\hline
\end{tabular}

\caption{The three possible conditions for a half adder.}
\label{fig:threehalfadders}
\end{center}
\end{figure}


\clearpage

\subsection*{Full Adder}

This type of adder is a little more difficult to implement than a half-adder. The main difference between a half-adder and a full-adder is that the full-adder has \emph{three} inputs and two outputs. The first two inputs are A and B, just like with every half-adder, and the third input is an input carry, designated as $C_{in}$. Enabling the circuit to handle a carry \emph{in} from another adder makes it possible to add many binary orders of magnitude, since many carry-outs and carry-ins will be needed to add numbers larger than 1. Once we implement a full adder, we can string eight of them together to create a byte-wide adder and cascade the carry bit from one adder to the next.
\bigskip

\begin{figure}[!hb]
\begin{center}
\begin{circuitikz}
\draw
% Inputs and outputs:
	(0,4.25) node[ocirc](ain) {} % A node
	(0,4.30) node[left] {{\color{red}$A$}} % A label
	
	(0,3.72) node[ocirc](bin) {} % B node
	(0,3.72) node[left] {{\color{red}$B$}} % B label

	(0,1.72) node[ocirc](cin) {} % C-in node
	(0,1.72) node[left] {{\color{red}$Carry_{in}$}} % C-in label

	(8.2,1.0) node[ocirc](cout) {} % C-out node
	(8.2,1.0) node[right] {{\color{red}$Carry_{out}$}} % C-out label

	(8.2,3.72) node[ocirc](sout) {} % S node
	(8.2,3.72) node[right] {{\color{red}$Sum$}} % S label
;

% Gates:
\draw
	(2.8,4.0) node[xor port](xorGate1) {} % xor gate 1
	(3.0,4.9) node[left] {$XOR_1$} % XOR_1 label

	(5.5,3.72) node[xor port](xorGate2) {} % xor gate 2
	(5.7,4.60) node[left] {$XOR_2$} % XOR_2 label 

	(2.8, 0.0) node[and port](andGate1) {} % AND gate 1
	(3.0,-1.0) node[left] {$AND_1$} % AND_1 label
 
	(5.5,2) node[and port](andGate2) {} % AND gate 2
	(5.7,1) node[left] {$AND_2$} % AND_2 label

	(7.75,1.00) node[or port](orGate1) {} % or gate
	(7.55,0.1) node[left] {$OR$} % OR label
;

% Tie A and B to XOR_1:
\draw
	(ain) to[short](xorGate1.in 1) 
;

\draw[green!70!black, thick]	
	(bin) to[short](xorGate1.in 2)  
;

% Tie XOR_1 output to XOR_2 input and to AND_2 input:
\draw[cyan, thick]
	(xorGate1.out) |- (xorGate2.in 1) 
	(3.5,4.0) node[circ, color=cyan]{}
	(3.5,4.0) |- (andGate2.in 1)
;

% Tie A and B to AND_1:	
\draw
	(0.5,4.25) node[circ] {}	
	(0.5,4.25) |- (andGate1.in 2)
;

\draw[green!70!black, thick]
	(1.0,3.72) node[circ, color=green!70!black] {}
	(1.0,3.72) |- (andGate1.in 1)
;

% Tie C-in to AND_2 and XOR_2:
\draw[brown, thick]
	(cin) |- (andGate2.in 2)
	(3.0,1.72) node[circ, color=brown]{}
	(3.0,1.72) |- (3.0, 3.4)
	(3.0,3.40) |- (xorGate2.in 2)
;

% Tie AND gate outputs to the OR gate:
\draw[orange, thick]
	(andGate2.out) |- (6,2)
	(6,2)   |- (orGate1.in 1)
;	

\draw[blue, thick]
	(andGate1.out) to[short](6,0)
	(6,0)   |- (orGate1.in 2)
;

% Outputs:
\draw
	(xorGate2.out) |- (sout) % XOR_2 to Sum
	(orGate1.out) |- (cout) % OR to Carry-out

;
\end{circuitikz}
\caption{A full-adder circuit, made up of two half-adder circuits. A full-adder enables carry-in as well as carry-out. This means that the full adder can add \emph{three} inputs together, and keep track of all of them. If the sum is two or three, the carry-out signal goes high. These units can be linked together to make binary addition work even for very large numbers.}
\end{center}
\end{figure}

Since full adders can accept three inputs, the truth table is pretty big.

\begin{center}
\begin{tabular}{ccc  cc  c}
\multicolumn{6}{c}{\textbf{Full Adder Truth Table}}\\
\hline\\[\negsep]
 &  & \textbf{Carry} &  & \textbf{Carry} & \\
\textbf{A} & \textbf{B} & \textbf{In} & \textbf{Sum} & \textbf{Out} & \textbf{Decimal}\\
\hline
0 & 0 & 0 & 0  & 0 & 0 \\
\grr
1 & 0 & 0 & 1  & 0 & 1 \\
0 & 1 & 0 & 1  & 0 & 1 \\
\grr
1 & 1 & 0 & 0  & 1 & 2 \\[\sep]
\hline\\[\negsep]
0 & 0 & 1 & 0  & 0 & 1 \\
\grr
1 & 0 & 1 & 0  & 1 & 2 \\
0 & 1 & 1 & 0  & 1 & 2 \\
\grr
1 & 1 & 1 & 1  & 1 & 3 \\

\hline
\end{tabular}
\end{center}

\begin{figure}[!ht]
\begin{center}
\includegraphics[scale=0.19]{FullAdderboard.png}
\caption{A single full-adder circuit for adding two bits plus a carry-in digit, implemented on a small circuit board using individual integrated circuit gates instead of discrete transistors. You can see the gates easily, but the board is much simpler to work with, compared to forming up a full adder from five of the single-gate boards.}
\end{center}
\end{figure}




\clearpage

\subsection*{Equality}

Comparing two one-bit numbers to see if they are identical is done with XNOR gates. These gates are the same as an XOR gate passing output into a NOT gate. 

\medskip
\begin{center}

\begin{tabular}{p{2.5in} p{3.5in} }
\hline\\[\negsep]

The symbol for an XNOR gate is:

\vspace{0.25in}

\begin{circuitikz}
\draw
	(0,0) node[xnor port](xnorGate) {}
	(xnorGate.in 1) node[left] {{\color{red}$INPUT~A$}}
	(xnorGate.in 2) node[left] {{\color{red}$INPUT~B$}}
	(xnorGate.out) node[right] {{\color{red}$OUT$}}
;
\end{circuitikz}

&

\centering

The ``Truth Table" (how they behave) is: 
\vspace{0.15in}

\begin{tabular}{ll | c}
\multicolumn{3}{c}{\textbf{XNOR Gate }}\\
\multicolumn{3}{c}{\textbf{Truth Table}}\\
\hline\\[\negsep]
\textbf{A} & \textbf{B} & \textbf{OUT}\\
\hline
0 & 0 & 1  \\
1 & 0 & 0  \\
0 & 1 & 0  \\
1 & 1 & 1  \\
\hline
\end{tabular}
\\
\tabularnewline

\hline\\[\negsep]

\end{tabular}
\end{center}

\bigskip

\begin{figure}[hb!]
\begin{center}
\includegraphics[scale=0.17]{XNOR1.jpg}
\caption{A discrete-transistor XNOR gate made up of one XOR and one NOT gate. Tying the output of the XOR gate to the input of the NOT gate makes the result of the NOT gate an XNOR of the original two XOR gate inputs---answering the question ``are the two inputs equal?"}
\end{center}
\end{figure}

To compare large numbers, the circuit requires an XNOR gate for each binary digit (bit) in the numbers being compared. If any XNOR gate returns zero, the numbers are not equivalent. To build all of the XNOR gates into a single yes or no answer, each XNOR output could go into a series of AND gates -- that is, since AND returns a ``yes'' only if both inputs are one, a single ``no'' prevents the AND gate from returning ``yes''. 


\begin{figure}[h!]
\begin{center}
\begin{circuitikz}

\draw

% Gates:
	(2,0.5) node[xnor port](xnorGate0) {} % xnor gate
	(1.05,1.05) node[above] {$XNOR_0$} % XNOR label

	(2,2.5) node[xnor port](xnorGate1) {} % xnor gate
	(1.05,3.1) node[above] {$XNOR_1$} % XNOR label
	
	(4.5,1.5) node[and port](andGate0) {} % AND gate 0
	(4.6,2.4) node[left] {$AND_0$} % AND label

	(2,6.5) node[xnor port](xnorGate3) {} % xnor gate
	(1.05,7.05) node[above] {$XNOR_3$} % XNOR label

	(2,4.5) node[xnor port](xnorGate2) {} % xnor gate
	(1.05,5.1) node[above] {$XNOR_2$} % XNOR label
	
	(4.5,5.5) node[and port](andGate1) {} % AND gate 1
	(4.7,6.4) node[left] {$AND_1$} % AND label
		
	(6.5,3.5) node[and port](andGate2) {} % AND gate 2
	(6.7,4.4) node[left] {$AND_2$} % AND label

% Input nodes:
	(0,0.77) node[ocirc](ainput0) {} % A node
	(0,0.83) node[left] {{\color{red}$A_0$}} % A label

	(0,0.25) node[ocirc](binput0) {} % B node
	(0,0.22) node[left] {{\color{red}$B_0$}} % B label

	(0,2.77) node[ocirc](ainput1) {} % A node
	(0,2.75) node[left] {{\color{red}$A_1$}} % A label

	(0,2.22) node[ocirc](binput1) {} % B node
	(0,2.20) node[left] {{\color{red}$B_1$}} % B label


	(0,4.77) node[ocirc](ainput2) {} % A node
	(0,4.72) node[left] {{\color{red}$A_2$}} % A label

	(0,4.22) node[ocirc](binput2) {} % B node
	(0,4.23) node[left] {{\color{red}$B_2$}} % B label

	(0,6.77) node[ocirc](ainput3) {} % A node
	(0,6.75) node[left] {{\color{red}$A_3$}} % B label

	(0,6.22) node[ocirc](binput3) {} % B node
	(0,6.20) node[left] {{\color{red}$B_3$}} % B label

% Output nodes:

	(7.5, 3.5) node[ocirc](output){}
	(7.7, 3.75) node[above] {{\color{red}$Result$}}
	(andGate2.out) |- (output)

% Nets:
	(ainput0) |- (xnorGate0.in 1) % xor A wiring
	(binput0) |- (xnorGate0.in 2) % xor B wiring

	(ainput1) |- (xnorGate1.in 1) % xor A wiring
	(binput1) |- (xnorGate1.in 2) % xor B wiring	

	(ainput2) |- (xnorGate2.in 1) % xor A wiring
	(binput2) |- (xnorGate2.in 2) % xor B wiring

	(ainput3) |- (xnorGate3.in 1) % xor A wiring
	(binput3) |- (xnorGate3.in 2) % xor B wiring	

	(xnorGate0.out) |- (andGate0.in 2)
	(xnorGate1.out)	|- (andGate0.in 1)

	(xnorGate2.out) |- (andGate1.in 2)
	(xnorGate3.out)	|- (andGate1.in 1)

	(andGate0.out) |- (andGate2.in 2)
	(andGate1.out) |- (andGate2.in 1)

;

\end{circuitikz}

\caption{A circuit to provide a yes/no answer to the question, ``are  two four-bit numbers, A and B, equal?" Each XNOR gate compares the bits in each position of each number, and passes the answer to an AND gate input. The AND results ripple forward to a final answer.}
\end{center}
\end{figure}


\begin{figure}[h!]
\begin{center}
\includegraphics[scale=0.16, clip=TRUE, trim=0 0 500 0]{4bitxnor.jpg}
\caption{Testing for equality with four XNOR gates. Here, XNOR gates are created with XOR gates that pass the output (green wires) to a NOT gate. An XOR gate only goes logic high if the inputs are different. Thus, NOT XOR means both inputs are the same! Each XNOR output (blue wires) passes to a NAND gate. Recall that NAND gates only return zero if both inputs are 1. The output of each NAND gate (purple wires) passes to one input of a NOR gate. So if both inputs to a given NAND gate are ``1'', the NAND gate will show ``0" as its result. NOR gates only return 1 if both inputs are zero. The result: the NOR gate lights up an LED if all inputs are equal. It is no different than the AND-gate cascade shown above in its result, it just uses negative logic to get there. Here, both inputs are equal to 1, so the LED is lit.}
\end{center}
\end{figure}







\clearpage


