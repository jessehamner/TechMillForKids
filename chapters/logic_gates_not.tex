\subsection*{The NOT Gate}

NOT gates, sometimes called \emph{inverters}, take one input and reverse the value of that input. So a 1 becomes a 0, or vice versa. Put another way, this gate creates the \emph{complement} of the input value. Notice that the symbol for a NOT gate, below, is the same as for the buffer, with a little bubble on the output. That bubble means ``opposite'' or ``NOT''. A buffer passes along the signal it is given; the NOT gate inverts the signal. So a 1 becomes 0, and a 0 becomes 1.

\medskip
\begin{center}

\begin{tabular}{p{2.3in} p{3.7in} }
\hline\\[\negsep]

The symbol for a NOT gate is:
\vspace{0.25in}

\begin{circuitikz}
	\draw(0,-0.5)
	node[american not port](notgate){}
	(notgate.in) node [left](){{\color{red}$INPUT$}}
	(notgate.out) node [right](){{\color{red}$OUT$}}
;
\end{circuitikz}

&

\centering

The ``Truth Table" (how they behave) is:
\vspace{0.15in}

\begin{tabular}{l | c}
\multicolumn{2}{c}{\textbf{NOT Gate }}\\
\multicolumn{2}{c}{\textbf{Truth Table}}\\
\hline\\[\negsep]
\textbf{IN} & \textbf{OUT}\\
\hline
0 & 1 \\
1 & 0  \\
\hline
\end{tabular}
\\
\tabularnewline

\hline\\[\negsep]

\end{tabular}
\end{center}

\bigskip

A simple NOT circuit, using a very old resistor-transistor logic (RTL) design. It is easy to understand, though this example is not how NOT gates are actually implemented any more, because of the poor efficiency---wasted energy turns into heat, which then has to be cooled somehow. It is better to not make the heat in the first place.

\begin{figure}[!ht]
\begin{center}
\begin{circuitikz}

\draw 
% IN node:
	(1,3) node[ocirc](in) {}
	(1,3) node[left] {{\color{red}$IN$}} % IN label

% OUT node:
	(6.0,4) node[ocirc](out){} 
	(6.25,4) node[right] {{\color{red}$OUT$}} % OUT label

	(4,4) node[circ](){}

% The Transistor:
	(4,3) node[npn](Q1){}
	(4,2.75) node[right]{$Q_1$}

% Vcc and GND:
	(4,6.5) node[vcc](vcc){$V_{cc}$}
    (4,2) node[ground](ground){}
    
% Resistors:
	(1,3) to [R, l^=$R_1$] (3.5,3)
	(4,6) to [R, l^=$R_2$ ](4,4.5)


% Nets:
	(Q1.E) to (ground)
	(Q1.C) to (4,4.5)
	(4,6) to (vcc)
	(4,4) to (out)

;
\end{circuitikz}

\caption{A very simple NOT gate schematic. When IN is logic low (off), electricity cannot pass through to the ground, and so it is available at the OUT terminal---OUT is a 1. When IN is carrying current, the transistor opens up and electricity starts flowing to ground -- the ``path of least resistance''. As electricity flows to ground instead of OUT, OUT has a very low voltage available -- making it a zero.}
\label{fig:simplenot}
\end{center}
\end{figure}


\begin{figure}[!ht]
\begin{center}
\begin{circuitikz}

\draw 
	(0,3.0) node[ocirc](in) {} %IN node
	(0,3.0) node[left] {{\color{red}$IN$}} % IN label
%	(0,3.0) |- (1,3.0)
	
	(7.5,4) node[ocirc](out){} % OUT node
	(7.75,4) node[right] {{\color{red}$OUT$}} % OUT label
	(6,4) |- (out)
	(6,4) node[circ](){}

% Vdd:
	(6,7.5) node[vcc](vin){}
    (5.5,7.5) node[above] {$V_{in}$} % Vin

% GND
    (6,0) node[ground](ground){}
 %   (6.75,0.5) node[below] {{\color{red}$GND$}} % GND

% P-type FET
	(6,5) node[pmos, emptycircle, xscale=1](Q1){}
	(6.5,5.0) node[above]{$Q_1$}

% N-type FET
	(6,3) node[nmos, rotate=0](Q2){}
	(6.5,2.5) node[above]{$Q_2$}

% Resistors:
	(4,0) to [R, l^=$R_1$](4,3)
	(Q2.S) to [R, l^=$R_2$ ](ground) 
	(in) to [R, l^=$R_3$](Q2.G) 
	(4,7) to [R, l^=$R_4$](4,5)

% nets:
	(vin) |- (Q1.S)
	(4,3.0) node[circ](){}
	(6,0.0) node[circ](){}
	(6,7) node[circ](){}
	(Q1.D) |- (Q2.D)
	(4,0) |- (ground)
	(6,7) |- (4,7)
	(4,5) |- (Q1.G) 
;
\end{circuitikz}


\caption{A practical inverter (NOT) circuit. While more complicated than a simple NOT circuit, it is more efficient, wasting much less energy. It uses one P-type and one N-type \emph{field effect transistor} (FET). P-type transistors clamp off electricity when their gate \emph{has} voltage on it. N-type transistors clamp off when their gate \emph{does not have} voltage on it. So, here, when the input is high, the top transistor clamps off voltage input, and the bottom transistor opens up, allowing any current to flow to ground--but there's no current available! The situation reverses when the input is low: the top transistor opens up, allowing current to flow, and the bottom transistor clamps off, preventing current from going anywhere -- but putting a high signal at the output.}
\label{fig:cmosnot}
\end{center}
\end{figure}


\begin{figure}[!ht]
\begin{center}
\includegraphics[scale=0.15]{trimmednotgate.png} % convert NOTgate.png -alpha off -trim  trimmednotgate.png
\caption{A circuit board designed using the more elegant NOT gate schematic seen in Figure \ref{fig:cmosnot}. The NOT gate takes exactly one \emph{input}, A, and produces a \emph{result}, R.}
\label{fig:notgateboard}
\end{center}
\end{figure}

\begin{table}
\begin{tabular}{lllll}
\hline\\[\negsep]
\textbf{Part} & \textbf{Value} & \textbf{Device} & \textbf{Package} & \textbf{Description} \\[\sep]
\hline\\[\negsep]
JP1  &  Input/Output & pin header    & 1x2 header     & Standard 2-pin 0.1" header \\[\sep]
JP2  &  $V_{in}$ \& $GND$ & pin header & 1x2 header     & Standard 2-pin 0.1" header \\[\sep]
Q1   &  ZVP3306A    & transistor        & TO-92-3     & P-type MOSFET transistor \\[\sep]
Q2   &  BS270       & transistor         & TO-92-3     & N-type MOSFET transistor \\[\sep]
R1   &  100K        & small resistor & 0204/5    & Pull-\emph{down} resistor \\[\sep]
R2   &  10K (10,000$\Omega$)         & Small resistor & 0204/5    & Current-limiting resistor \\[\sep]
R3   &  100R (100$\Omega$) & $\frac{1}{4}$-watt resistor & 0207/7 & Current-limiting resistor \\[\sep]

\hline\\
\end{tabular}

\caption{The list of components needed to make the NOT gate seen in Figures \ref{fig:cmosnot} and \ref{fig:notgateboard}.}

\end{table}
